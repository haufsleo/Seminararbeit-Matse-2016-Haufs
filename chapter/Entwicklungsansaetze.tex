\chapter{Entwicklungsansätze mobiler Applikationen}
\label{chap:entwicklungsansaetze}
% Ansatz: Summe ca. 4 Seiten
%
% weitere Infos:
% http://www.uptopcorp.com/post/mobile-app-development-native-vs-hybrid-vs-mobile-websites
%
Im laufe der letzten Jahre haben sich verschiedene grundsätzliche Ansätze für die Entwicklung mobiler Applikationen durchgesetzt. Bei der Wahl eines dieser Ansätze müssen verschiedene Anforderungen an die mobile Applikation betrachtet werden. Zunächst muss entschieden werden, welches Betriebssystem die App unterstützen soll. Dabei haben sich im laufe der letzten Jahre die Betriebssysteme iOS und Android durchgesetzt. Im dritten Quartal 2016 hatte iOS einen weltweiten Marktanteil von 12.1~\% und Android einen Marktanteil von 87.5~\% \cite{strategyAnalyticsMarktanteile}. Da die beiden Betriebssysteme somit einen gemeinsamen weltweiten Marktanteil von 99,6~\%  bilden, spielen die weiteren Systeme Windows Phone und Blackberry OS nur eine sehr untergeordnete Rolle.

Eines der Ziele einer mobilen Applikation ist es, eine für die Plattform typische Benutzererfahrung zu schaffen und keine großen Irritationen zu verursachen. Dies muss natürlich bei der Umsetzung beachtet werden \cite{FrancisHybridApp}.

Zur Entwicklung mobiler Applikationen gibt es grundsätzlich drei verschiedene Ansätze, die sich folgendermaßen charakterisieren lassen:
\begin{itemize}
    \item Native Entwicklung der Applikation in der Programmiersprache, die ursprünglich für die Entwicklung auf dieser Plattform vorgesehen war
    \item Systemübergreifende Entwicklung mittels einer Web-Applikation
    \item Systemübergreifende Entwicklung mittels hybrider Frameworks
\end{itemize}
%
% --------------------------------------------------------------------------------------------
%
\section{Native Applikationen}
\label{sec:nativeApplikationen}
%
Unter nativer Anwendungsentwicklung wird die direkte Entwicklung für eine Plattform in der jeweiligen ursprünglichen Programmiersprache verstanden. Bei iOS ist die native Programmiersprache Objective-C bzw. zunehmend Swift \cite{appleDokuSwift}. Für Android wird nativ in Java programmiert \cite{googleAndroidDoku}. Ein großer Vorteil nativer Applikationen ist die hohe performanz. Native Apps sind für ein bestimmtes Betriebssystem optimiert und daher für komplexe und rechenintensive Anwendungen sehr gut geeignet. Objective-C ist beispielsweise zudem als Erweiterung von C eine sehr maschinennahe Sprache \cite{appleObjectiveC}. 

Da iOS und Android als die beiden Hauptplattformen für mobile Applikationen zwei verschiedene native Programmiersprachen haben müssen bei einem nativen Entwicklungsansatz auch zwei getrennte Anwendungen entwickelt werden. Dies führt zu einem großen Mehraufwand in der Softwareentwicklung.

Neben der nativen Programmiersprache liefern die Plattformen auch hauseigene Entwicklungswerkzeuge. Für die iOS-Plattform gibt es die Entwicklungsumgebung XCode, für Android wird das Android Studio bereitgestellt. Neben der nativen Programmiersprache spielen die Frameworks und Programmierschnittstellen, die APIs (\enquote{Application Programming Interface}) eine wichtige Rolle, da diese es erlauben, Zugriff auf die Hardwarefunktionalitäten des Gerätes zu nehmen. So kann mit wenig Aufwand Zugriff auf Kamera, GPS-Modul oder die verschiedenen Sensoren des mobilen Endgerätes zugegriffen werden. Zudem kann je nach System Zugriff auf die geschützten Daten des Gerätes wie Kalender oder Adressbuch des Nutzers genommen werden. 

Nativ entwickelte Apps lassen sich im jeweiligen App Store veröffentlichen und vertreiben. Zudem regelt der App Store die Möglichkeit Updates der Software veröffentlichen zu können. Apple unterzieht dabei die auf ihrer Plattform zu veröffentlichende Software einem Prüfverfahren, um enge Qualitätsstandards zu gewährleisten. Apple unterzieht dabei jede App einem Review-Prozess und lässt keine Apps im App Store zu, die beispielsweise plötzliche Abstürze, elementare Verstöße gegen die User Interface-Guidelines von Apple, kaum reale Funktionalität und damit keinen Mehrwert für den Nutzer oder inhaltliche Bedenken wie Verletzung gegen Jugendschutzvorgaben haben \cite{appleReview}.
%
\subsection{Vorteile}
%
Durch die Veröffentlichungsmöglichkeit in App Stores wird Entwicklern eine direkte Vertriebsmöglichkeit der eigenen Software geboten. Durch die Qualitätskontrollen im App Store werden Applikationen mit schwerwiegenden Mängeln aussortiert und Anwendern ein gewisser Qualitätsanspruch für die im App Store angebotene Software garantiert.

Die nativen Frameworks schaffen eine plattformspezifische Nutzungserfahrung, die von reinen Webseiten oftmals nicht erreicht werden kann. Sie liefern dem Nutzer zudem die Möglichkeit auf Bereiche der mobilen Plattform oder Sensoren der Hardware zuzugreifen, auf die Web-Applikationen keinen Zugriff haben.

Da native Programme die volle Hardwareleistung des Gerätes nutzen können ermöglichen diese weit rechenintensivere Anwendungen. So können 3D-Spiele oder Simulationen berechnet werden, die in einer Web-App oder Hybriden App nicht möglich wären.
%
\subsection{Nachteile}
%
Bei nativer Softwareentwicklung muss für jede Plattform eine eigene Anwendung programmiert werden, was die Entwicklungskosten massiv erhöht. Da jede Plattform eigene Frameworks verwendet, die sich zum Teil gravierend in ihrer Funktionsweise unterscheiden ist es für Anbieter mehrerer getrennt entwickelter Apps nur sehr schwer eine gleichwertige Benutzererfahrung aller Systeme zu ermöglichen. So wird es vorkommen, dass manche Funktionen nicht auf allen Plattformen zur Verfügung steht.

Bei einer Aktualisierung der App muss stets ein Update über den App Store erfolgen, welche nur durch den Nutzer initialisiert werden können. Es ist also nicht zwingend sichergestellt, dass immer die aktuelle Version auf dem Gerät des Nutzers installiert ist.

Die Entwicklung nativer iOS Apps ist zudem mit dem Nachteil verbunden, dass diese nur auf Macs möglich ist, da nur hier der Code kompiliert werden kann. Die Veröffentlichung der Anwendung in den App Stores ist zudem mit einer Mitgliedschaft im Entwicklerportal verbunden, die bei Apple 99,-~\$ pro Jahr kostet. 
%
% --------------------------------------------------------------------------------------------
%
\section{Mobile Web-Applikationen}
\label{mobileWebApplikationen}
%
Unter einer mobilen Web-Applikation wird eine Applikation verstandn, die ausschließlich im Browser des Smartphones ausgeführt wird. Die Anwendung wird dabei mit allen benötigten Daten wie HTML-Dateien, CSS-Stylesheets oder Bildern von einem Webserver geladen und mittels HTTP-Protokoll an den Client übertragen. Im Gegensatz zu statischen Websites sind Webapplikationen dynamisch und erlauben eine Interaktion des Nutzers der Seite. Ermöglicht wird dies durch den Einsatz von Skriptsprachen wie JavaScript. 

Mobile Webapplikationen sind an die Anforderungen einer mobilen Plattform angepasst und opimiert. So ermöglicht der Einsatz von \emph{Responsive Webdesign}, dass sich das Erscheinungsbild (Struktur der Seite und das Styling) einer Webseite dynamisch an die Bildschirmgröße, das verwendete Betriebssystem, die verwendete Auflösung und die Ausrichtung des Gerätes anpasst. Zudem sind Eingaben auf Events wie \enquote{Touch}, \enquote{Swipe} oder \enquote{Pinch} optimiert. Im Idealfall wird zudem das Design (CSS-Bibliotheken) und das Verhalten der dynamischen Website soweit angepasst, dass die Webseite kaum von einer nativen Applikation unterschieden werden kann. 

Neben der Möglichkeit die Webseite über den Browser aufzurufen bietet iOS zusätzlich die Möglichkeit, ein Lesezeichen der Web-App auf dem Startbildschirm zu hinterlegen. So entsteht verstärkt der Eindruck einer wirklichen App, obwohl in Wirklichkeit nur ein Browserfenster in Vollbildmodus geöffnet wird \cite{safariRef}.
%
\subsection{Vorteile}
%
Durch eine völlig plattformunabhängige Entwicklung kann derselbe Code auf allen Plattformen ausgeliefert werden. Somit werden mit vergleichsweise geringerem Entwicklungsaufwand als bei nativer Entwicklung mehr Nutzer erreicht werden. Zudem kann Code aus Desktop-Webanwendungen nach einer Anpassung auf die mobile Webanwendung übertragen werden. Da die Distribution der App hier vollkommen vom Server des Herstellers und unabhängig von einem App Store kommt ist auch sichergestellt, dass immer die neueste Version auf dem Gerät des Anwenders angezeigt wird. Zudem ist die Veröffentlichung unabhängig von Richtlinien des App Store-Betreibers und deren Provision am Umsatz. Zwar ist die grundsätzliche Ansteuerung von Hardwarekomponenten des Smartphones aus dem Webbrowser nicht möglich, jedoch ermöglichen moderne Browser nach Erlaubnis durch den Nutzer die Abfrage der Geoposition \cite{browserGPS}.
%
\subsection{Nachteile}
%
Der gravierende Nachteil von Webapplikationen ist die nicht ausreichende Unterstützung von Hardwareschnittstellen. So können Kamera, Adressbuch oder Kalender nicht angesprochen werden. Speicher, der bei anspruchsvolleren Anwendungen nötig ist, kann zudem nicht in beliebiger Menge und auch von Plattform zu Plattform in unterschiedlicher Menge in Anspruch genommen werden. So können in der App generierte Daten nicht in beliebiger Menge persistiert werden. Zudem ist die persistierung von Daten von der Umgebung des Browsers abhängig. Wird der Cache des Browsers geleert sind die Daten ebenfalls gelöscht. Der Zugriff auf die nativen Bibliotheken des Smartphones sind ebenfalls nicht möglich.
%
% --------------------------------------------------------------------------------------------
%
\section{Hybride Applikationen}
\label{sec:HybrideApplikationen}
%
Hybride Applikationen sind grundsätzlich native Anwendungen. Sie bestehen hauptsächlich aus einer Webview, also einem Browser-Steuerelement, welches zudem über eine Schnittstelle zu nativen Modulen verfügt. Mithilfe dieser Module können in der Webview gestartete Web-Applikationen native Hardware-Funktionalitäten des Betriebssystems ansprechen über die allgemeine Browser nicht verfügen.

In iOS kann dies beispielsweise mittels einem Steuerelement der Klasse \texttt{UIWebView} \cite{uiWebView} umgesetzt werden, das es ermöglicht HTML-Inhalte einer lokal gespeicherten Webseite innerhalb einer nativen App darzustellen und als Container um die Web-App dient. Eine so dargestellte Webseite verfügt über alle im Browser zur Verfügung stehenden Technologien. Neben der in der Webview geladenen Seite besteht eine Hybride App jedoch aus Schnittstellen zu nativen Modulen, die aus der Webview, also aus der Webseite heraus, angesprochen werden können. So können je nach verwendetem Framework Module zur Verfügung gestellt werden, die die native Funktionalitäten des Smartphones ansprechen. Somit ist es beispielsweise möglich die Sensoren des Gerätes, die Kamera, das Adressbuch oder den Kalender anzusprechen. 

Eine der größten Entwicklungsplattformen für Hybride Apps ist Cordova, auf die in Kapitel \fref{sec:ApacheCordova} näher eingegangen wird.  
%
\subsection{Vorteile}
Hybride Applikationen vereinen die Vorteile von nativen, als auch von Web-Apps in sich. Die Entwicklung lässt sich plattformunabhängig gestallten. Grundsätzlich wird für alle Plattformen eine Web-App erstellt, die anschließend in eine Container-App integriert wird. Werden native Funktionalitäten erfordert können plattformspezifische Module (Plugins) verwendet werden.

Durch die Einbindung solcher Module können beliebige Bibliotheken des Betriebssystem nativ genutzt werden. Dies führt theoretisch zu einem uneingeschränkten Hardwarezugriff (im Rahmen der Freigabe durch das Betriebssystem) und somit zu einer nativen Nutzererfahrung. 

Im Gegensatz zu einer reinen Web-App können die Kerninhalte der Applikation bei Auslieferung bereits mitgeliefert werden oder einmalig heruntergeladen werden und im für native Anwendungen vorgesehenen Dokumentenspeicher abgelegt werden können. Somit ist eine dauerhafte persistierung von Daten möglich und eine Verringerung des Datenverbrauchs.

Wie bei Webanwendungen ist bei hybriden Anwendungen zudem der Umsatand gegeben, dass Entwickler aus dem Bereich der Webentwicklung an der Entwicklung der hybriden App mitarbeiten können, da die verwendeten Technologien grundsätzlich die gleichen sind wie bei allgemeiner Webentwicklung. 
%
\subsection{Nachteile}
%
Hybride Anwendungen haben jedoch den Nachteil, dass die Entwicklung aufgrund der zu treffenden plattformabhängigen Fallunterscheidung beim Einsatz von Plugins aufwändiger ist als die reine Entwicklung von Webapps. Die Funktionsweise der Plugins kann sich eventuell je nach Implementierung unterscheiden. Zudem sind native Funktionalitäten nur möglich, wenn ein entsprechendes Plugin existiert. So entsteht eine gewisse Abhängigkeit von den Entwicklern nativer Plugins, wenn diese nicht selbst entwickelt werden können. Im Gegensatz zu nativer Entwicklung, bei der sofort die neuesten Technologien der Frameworks verwendet werden können müssen bei hybrider Entwicklung zunächst entsprechende Plugins entwickelt werden.

Einige der Nachteile nativer Appentwicklung werden ebenfalls mit übernommen. So muss bei jeder Aktualisierung der App ein Update in den App Store geladen werden. Es kann also nicht sichergestellt werden, dass die Endanwender immer die neueste Version der App verwenden. Zudem muss der Review-Prozess des App Stores durchlaufen werden. 