\chapter{Einleitung}
%
Der Markt für mobile Smartphones vergrößerte sich in den letzten Jahren enorm\cite{ChaffeyMarketStatisticsUsageAndAdoption}. In ähnlichem Maße ist der Markt für deren Apps in den letzten Jahren gewachsen. Von etwa 7 Milliarden~\$, die im Jahr 2011 weltweit in diesem Marktsegment umgesetzt wurden wird für 2016 von etwa 27,6 Milliarden~\$ weltweitem Umsatz ausgegangen \cite{StatistaGlobalRvenues}.

Neben dem Markt für mobile Applikationen hat sich auch der Bereich der Webentwicklung durch neue Frameworks und Engines in den Webbrowsern in den letzten Jahren stark weiterentwickelt. Viele Anwendungen, die früher nur nativ auf einem mobilen Gerät möglich waren lassen sich heute mittels Webtechnologien (HTML5, CSS und JavaScript) verwirklichen.

Da heute mit Android und iOS zwei große marktbeherrschende mobile Plattformen existieren, sind heutige Entwickler gezwungen bei einer gewünscht hohen Marktabdeckung ihrer Software und dem klassischen Ansatz einer nativen Softwareentwicklung mehrere getrennte Anwendungen in verschiedenen Programmiersprachen zu entwickeln, was zu großen Zusatzkosten führt. Hier setzt die Kombination von Webentwicklung und mobiler Softwareentwicklung in Form von hybriden Smartphone-Apps an.
\\
\\
Im Rahmen dieser Arbeit werden mögliche Ansätze mobiler Softwarentwicklung verglichen und insbesondere der Bereich der hybriden Softwareentwicklung mit dem Ionic 2-Framework vorgestellt. Zudem wird die Erweiterung des Funktionsumfanges des Ionic Frameworks mithilfe von nativen Komponenten beschrieben. Native Komponenten sind über eine Schnittstelle angesprochene in der originalen Programmiersprache entwickelte Module, die in bestehende hybride Projekte eingebunden werden können. Diese Erweiterbarkeit des Ionic-Frameworks wird exemplarisch am Beispiel der Kommunikation einer mit Ionic entwickelten iOS App mit einer zugehörigen Apple Watch App gezeigt.