\chapter{Einleitung}
%
Der Markt für mobile Smartphones vergrößerte sich in den letzten Jahren enorm. In ähnlichem Maße ist der Markt für Apps gewachsen. 

Neben dem Markt für mobile Applikationen ist auch der Bereich der Webentwicklung in den letzten Jahren enorm gewachsen und die verwendeten Frameworks und Engines in den Webbrowsern haben sich stark weiter entwickelt. Viele Anwendungen, die früher nur nativ auf einem Gerät möglich waren lassen sich heute mittels Webtechnologien verwirklichen.

Da es heute mehrere führende mobile Plattformen gibt, sind heutige Entwickler gezwungen bei einer gewünscht hohen Marktabdeckung und einem Ansatz der nativen Softwareentwicklung mehrere getrennte Anwendungen zu entwickeln. Hier setzt die Kombination von Webentwicklung und mobiler Softwareentwicklung an.
\\
\\
Im Rahmen dieser Arbeit sollen die möglichen Ansätze mobiler Softwarentwicklung vorgestellt werden und insbesondere der Bereich der hybriden Entwicklung mit dem Ionic Framework vorgestellt werden. Zudem wird die Unterstützung nativer Komponenten am Beispiel der Anbindung einer mit Ionic entwickelten iOS App an eine zugehörige Apple Watch App gezeigt.